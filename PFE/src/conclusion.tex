To summarize, during the course of the project, I have taken part in adapting an MVP
for logistics, specifically the management of trucks and their drivers within the 
warehouse with the organisation MMSoft, to SaaS standards and turn it to a competitive
product within the market. The initial solution was more a proof of concept with the 
capability for real usage, but missing the security, scalability and maintainability
requirements. So we undertook a full-fledged project with the following goals:
    \begin{itemize}
        \item Implement archival capabilities to store the data in a database.
        \item Implement security features, such as rate limiting and update the authorization system.
        \item Migrate the product to a scalable and rigid infrastructure, with automated scaling.
        \item Adding Continuous Integration and Continuous Delivery to multiple levels.
        \item Refactoring the code to make it more readable and maintainable.
    \end{itemize}

The project was took up by a team of three people, handling the tasks with
the kanban method, which had five states:
    \begin{itemize}
        \item Stand-by
        \item Request
        \item In progress
        \item To Review
        \item Done
    \end{itemize}

The implemetation of the project for multiple of the goals has been a success, for
the most part. So far, the archival has been implemented within the AWS database
DynamoDB, and S3 for the files archival. The security has took turn into moving 
the JWT Bearer token transfer to a cookie based system, where the token is stored
in the browser as a HTTP Only cookie, then within the same subject we had API 
rate limiting, which was implemented using Bucket4J library within the filter layer
of the API gateway. The infrastructure took a transfer from the EC2 instance to 
a ECS cluster, which allows for better scalability and maintenance for the running 
system. The CI/CD has been implemented using the Circle CI hosted service,
with the help of quality gateways offered by Sonarqube to assess the quality of the
code to allow it to pass or not.
The tests have been implemented using the JUnit and Mockito libraries, and the
implementation of the API has been done using the Spring Framework.
For each step or task of the project, we were implementing unit tests or ensuring
functionality of the task with manual tests, before moving it to the "To review" state
within the kanban board.
The reviewing of the project was done by the product owner, who is the CEO and 
knows how exactly the product should be implemented for the clients needs.

Of course, the project was not completed to a 100\% in the time allotted, so the
limits reside currently in the scope of the performance of some endpoints 
that rely on pulling big sets of data from the Redis Database and which is currently
being investigated for a better implementation. The refactoring part of the project
is still heavily in progress, as there are still many decisions to be made regarding
the technologies to use and the way to implement them before diving into it.

So for the future of the project, it would be clear that targetting the refactoring
of the code base would be the best way to improve the project, as it would open 
doors for easier implementation of the new features and have some significant improvement
in the performance of some services.
