With the general structure of the software present, it is clear that the
software is missing some important features and building blocks that can make it 
the SaaS it's aimed to be.

To start with the essentials, as the software is targeted to be used for logistics
within factories and warehouses, it's important for the data of daily interactions and
operations to be stored in a database and be accessible to analytics and KPI's operations
to be able to provide a better understanding of their efficiency and how their
work is flowing. Because for the mean time, all the daily operations data is being stored
in the Redis database and Redis being an in-memory data structure store used as a database,
message broker, and streaming engine \cite{redis_def} it won't be suited for long term storage
as to keep it operational it will need to clear old data and free up space, thus why it's cleared
on a daily to weekly basis. And with no archival solution put in place, the data for now is
cleared then lost which means for any client opting to pull their old data for analytics
won't be possible.

Next comes the user management system, which is clearly a key factor in such a workflow, 
as seen in the \ref{fig:flowchart} there are multiple actors involved in the process,
from the gate keepers, operators, drivers, and the management staff. So with the current
user system which opts to have user set as a site which means for each factory there's a
single user or even though you can make multiple users they all will have full permission
in the site's scope which isn't secure neither is a good idea. So opting to a new system
which encourages the use of a role system for users and the users then belonging to a site
would be a better solution, as it would even open the flexibiltiy of adding configuration
to the user roles, and have each user have their own scope of permissions depending on
their role.

Also for the authentication, the current system uses a Bearer token which is a Json Web
Token (JWT) and that is permanent rather than an ephemeral one. So if an attacker manages
to intercept a token, which in fact is doable with code injections to expose a request,
then they can use it to access the backend at any time.
Opting to some other authentication system, such as a cookie based system, would be more
secure, as the cookies could at least be held in a secure place such as code injections,
can be used to acquire them from the front-end.

Looking at the current system and how the deployment is done, which is done in a manual way,
so it's takes more time to deploy simple bug fixes and new features reducing from the efficiency
of the developers, due to the time it takes to deploy each time. For fact, it's clear that
the addition of some automized deployment tools would be a good way reduce the need for the
manual deployment and also allow for faster deployment of new features and hot-fixes as the
deployment process is mostly a recurrent set of actions.

Next comes the code quality, which is rather messy, as it uses a lot of different tools,
some part are just redundant which could be done in a DRY\footnote{DRY or don't repeat
yourself is a term used to describe the practice of writing code that is not repetitive.
It is a way to reduce the amount of code that needs to be written.\cite{pragmatic_bro}} way,
not to the extreme of course where the code is becoming DRY but in expense of readability
and complexity. Other than the redundancy of some parts, the use of different approaches
in the code can get kind of messy and hard to maintain, especially if you're not familiar
with the code which would be the case in the expansion of the team, so the more the code
is unified the better. Also the code is not documented, so it's not clear what some parts
of the code internals do when it's something complex and that requires a modification.
The tests for the majority of the code are not put in place, so in the case of huge update
to the code, it would not be possible if some behaviour changed or if some parts are just 
not working.

There's the usage of different tech stacks which are NodeJS and Java Spring, which is a bit
of a mess, as it's not clear why the NodeJS is being used in the backend without much impact
as it's just handling some simple backend logic that could migrated to a Java Virtual
Machine(JVM)\footnote{JVM is the virtual machine that's capable of running Java binary
code}, compatible language such as Java, Kotlin, Scala, \dots, why I said a JVM compatible
language it's because the majority of the system is built on Java so it would make it
easier to migrate Js to Java than let's say moving the Java to NodeJS, but also to gain
the advantages of strongly typed language making it easier to pick up bugs. With such a
move, it would create a more uniform and rigid way to have the whole project fully handled
with single build tool, which in turn can really ease up the deployment process, and make
it automatable more easily.

One other point to tackle is security within the code, the current code base uses secrets
which are stored directly in the code, which is not secure, as you can easily acquire the
original code through a decompiling tool, or even through a reverse engineering tools,
which would make those secrets attainable directly. Opting for some other way of storing
the secrets, such as a database in addition with environment variables, would probably be
a better solution, in addition with a rolling system of secrets in possible cases making
those keys temporary.

\section{Conclusion}

To summarize the current state of the software, it's clear that the current application
is at in terms of the features and the current state of the code, it's clear that there's
a strong foundation for the application to be a SaaS as it's already handling almost the
full workflow of the fleet management system, , but there are some missing features and
building blocks specifically in the infrastructure and security aspects, which could be
mediated to complete the organization view of their product.
