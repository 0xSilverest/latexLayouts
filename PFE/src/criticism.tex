This already existing solution comes with its own set of problems.

Beginning from the most obvious one we have at hand is the usage of different technologies
in the backend without any advantage in place for opting to this approach, such as NodeJs
and Spring Boot, which in the long term affects the maintainability of the solution, as
reported by the CTO, was developed by a freelancer and went untouched since. So rather
than a maintainable solution, they opted for a momentary solution to have a PoC which
could assess that the product is working as intended.
While it's only handling the registration process, it would be better if it just migrated
to the same stack of the whole system. That will allow the reusability of code,
as there are already implementations for handling the Users Database on the Spring code
base, and there is really no purpose to separate the systems.

For the databases as of the current solution, there are already two databases put in place
a MySQL database to deal with any data that relates to users, the other is a Redis database
to deal with real-time data such as tracking truck entries etc. Redis being the database to
deal with real-time data, but the problem with redis is that it's load everything in memory
so it's can be overloaded with costs of having memory coming at a great expense for some
data that won't be casually needed. So a archival system for the data, where cost of the
storage comes cheaper but, has low throughput for reads and writes would be perfect for
such usage, the archival database would only take daily backups as afterwards this data
would mostly be used either for KPIs, and thus the Redis database would get cleaned on
a daily basis not having extra data taking out memory.

The current deployment of the solution is not scalable neither seamless, as it 
relies heavily on tasks done by hand, not some automated process.
So the there are no tests for updating the system, pipelines for deployment, etc. Which
means if there is a big update to the application, it would be more susceptible to
downtime due to some bugs or other issues, that were not caught beforehand.

Furthermore, going through the existing code base, the secrets that relate to anything
that is credentials or encoding keys are all exposed in the code, in terms of
industrialization, it means there's a lack of security.

Also, there are some issues relating to missing  unit tests,
so the tests are not covering all the features but just some some of them,
which is a problem for the maintainability of the code also for adding new features.

Staying in the current point, which is the code base, there is a lack of documentation
and the code is not well documented, so it's not easy to understand what is going on.

Finally, some code could use some refactoring, as it's not really well organized,
and has reused code chunks that could be made simpler if just extracted to some function 
to be valled within the needed spots reducing the redundancy.
