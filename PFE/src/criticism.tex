This already existing solution comes with its own set of problems.

Beginning from the most obvious one we have at hand is the usage of different technologies in
the backend without any advantage in place for opting to this approach, such as NodeJs and Spring
Boot, which in the long term affects the maintainability of the solution, as reported by the CTO,
was developed by a freelancer and went untouched since. So rather than a maintainable solution,
they opted for a momentary solution to have a PoC.
While it's only handling the registration process, it would be better if it just migrated to the
same stack of the whole system. That will allow the reusability of code, as there are already
implementations for handling the Users Database on the Spring code base, and there is really
no purpose to separate the systems.

For the databases as of the current solution,
there are already two databases put in place a MySQL database to deal with any data that relates to users,
the other is a Redis database to deal with real-time data such as tracking truck entries etc.
Redis being the database to deal with real-time data, but the problem with redis is that it's load everything
in memory so it's can be overloaded with costs of having memory coming at a great expense for some data
that won't be casually needed. So a archival system for the data, where cost of the storage comes cheaper but,
has low throughput for reads and writes would be perfect for such usage, the archival database would only take
daily backups as afterwards this data would mostly be used either for KPIs, and thus the Redis database would
get cleaned on a daily basis not having extra data taking out memory.

Furthermore, going through the existing code base, the secrets that relate to anything that is credentials or
encoding keys are all exposed in the code, in terms of industrialization, it means there's a lack of security.

Lastly, the current deployment of the solution is not scalable, as it's was mostly developed as a PoC,
so the there are no heavy tests for updating the system, pipelines for deployment, etc. So if there is a
big update to the application, it would be more susceptible to have downtime due to bugs or other issues,
that were not caught beforehand.

% TODO: Improve on the criticism put in place
