La logistique est un domaine o\`u la technologies et la digitalisation ont conduit 
l'innovation dans ce domaine. C'est dû \`a la croissance rapide de cette industrie.
Donc comme c'est un march\'e o\`u l'innovation peu prendre \c{c}a place et que les 
nouveau projets ont pu prendre leur part. C'est pour cette raison o\`u des solutions
MVP sont cr\'ees pour bien comprendre les besoins des utilisateurs et bien s'adapter
au domaine au fur et \`a mesure de la cr\'eation de ces solutions. Donc durant mon 
projet PFE j'ai int\'egr\'e le sujet de l'indistrualisation d'un MVP, c'est \`a dire
d'adapter un MVP aux normes des SaaS. C'est \`a dire transformer le MVP en un SaaS qui
offre plusieurs avantages pour le cycle de vie de la cr\'eation et d\'eveloppement de
la solution, et l'exp\'erience des utilisateurs.

Durant la r\'ealisation de mon projet PFE, j'ai pris le r\'ole de suivre les diff\'erentes
t\^aches, allant d'impl\'ementation des bases de donn\'ees pour la conservation des
donn\'ee et l'archivage. Int\'egration des normes de s\'ecurit\'e au niveau des filtres
de l'application. Et puis, nous avons migr\'e l'infrastructure d'une  machine virtuelle
linux vers des images Docker d\'eployer sur des contenaires AWS. Et Finalement, nous
avons ajout\'e des fonctionnalit\'es pour la gestion des utilisateurs.

\motcles{
    Gestion de flotte, MVP, SaaS, Standard industriel}
