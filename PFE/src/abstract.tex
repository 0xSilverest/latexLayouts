The logistics has been one of the major industries that's oriented to the
adoption of technology and digitalization for simplifying the work in their overly
complex workflows and helps with their decision making, leading to a overall fast
growth pace. Thus there have been a swarm of new ideas trying to acquire their
share of the market, leading to innovative idea and as we say new sought
solutions, these solutions come mostly as proof of concepts(PoC) or minimum viable products(MVP) as they are cheaper
to produce, and offer a better possbility of creating a product that's more adept to the
needs of the end users. During my graduation project, I undertook the subject of adapting
an MVP to industrial standards, meaning that we transform the MVP into a Software as a
Service(SaaS) which provides multiple advantages for the health of the development of the
product, and the experience of the users.

During the implementation and contribution to the project, I took over the following
tasks.
Persistence of the data specifically archival.
integrating security on the code level as filters to undesired access or spams.
Migrating infrastructure to a containerized environment for better availablity and
scalability.
Implementing deployment strategies within continuous integration and delivery.
Adding new features to the solution that were much needed for the users.

These subjects were the main ones that I undertook during this project, where
we interacted with various tools and technologies, such as Git, AWS toolkit, Docker and
a backend fully focused on Java Spring.

\keywords{
    Minimum Viable Product,
    Software as a Service,
    Fleet management,
    Industrial standards}
