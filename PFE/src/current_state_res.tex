In this following chapter, we'll be looking at the current state of the whole system.
Taking a look at the current business process, which is constituted of the following
core activities:
\begin{itemize}
    \item Driver handling from accessing the warehouse to leaving it
    \item Digitalization of documents
    \item Secure access to logistics platform
    \item Dashboard for the management with the following functionalities:
        \begin{itemize}
            \item Truck status
            \item Driver status
            \item Document status
            \item Notification system
            \item Guidance system
        \end{itemize}
\end{itemize}

Also we'll be taking a look at the core services provided by the system which are
currently made of four front-end services a web-interface, a mobile application and
a IOT oriented front. Four back-end services a MQTT broker for communication, orchestration
service for real-time data management, registration service for the drivers and trucks,
and a API exposer for integration and last two databases one for handling data storage and the other for handling realtime-data.

Currently the main technologies used are React, Angular and Cermate for front-end,
and Spring, NodeJS for backend modules and all of this is hosted within the hosted
services of AWS.
\newpage

\section{Introduction}

As the solution is already an existing one, we'll be diving into the current state of
the system. Taking a look at the business process, then moving to the core services
explaining the goal of each of them. Then we'll be looking at the current state of the
backend architecture with some overview around the databases and technologies used, and 
why they are used.

Then on the second part of the chapter we'll be taking into account the already existing
system, to build a better understanding of it choices and have some constructive criticism
to take a look at parts that could be improved.

